\sum_{n = 1}^{\infty}  \chapter{分治法之大数乘法}
\begin{introduction}
	\item 问题背景
	\item 直接分治法
	\item 改进分治法
\end{introduction}

\section{问题描述}
给定两个大数$A$和$B$, 试计算
\begin{math}
	A \times B
\end{math}.
其中$A$和$B$分别表示为
\begin{math}
	A = a_n a_{n-1} a_{n-2} \ldots a_2 a_1
\end{math}
,
\begin{math}
	B = b_n b_{n-1} b_{n-2} \ldots b_2 b_1
\end{math}.
根据已学知识,给出如下引理。

\begin{lemma}{}{label_for_a+b}
	直接计算$A + B$,其复杂度为$O(n)$, 其中$n$为$A$和$B$的十进制位数。
\end{lemma}

直接计算$A \times B$时,我们将$A$与$B$的各位相乘,在将各中间结果相加,得到最终结果。
不难得出,这一过程需要进行$n$次基本乘法与$n+1$次加法。
根据引理\ref{lem:label_for_a+b},有:
\begin{theorem}{}{label_for_a*b}
	直接计算$A \times B$的时间复杂度为$O(n^2)$.
\end{theorem}

由定理\ref{thm:label_for_a*b}和引理\ref{lem:label_for_a+b}可知,如果我们直接相乘两个大数,其时间复杂度相比加法运算高出一个量级。
由于乘法在计算机中大量存在,我们希望找到更好的算法来降低乘法计算的时间复杂度,以提升计算机的性能。
分治法为我们提供了一条途径。
\section{直接分治法}
\subsection{算法描述}
这是一种简单的分治方法,将两个大数分为前后两部分,进行相乘。不失一般性,这里假设$n$为偶数。
将$A$与$B$分割为$A_2$, $A_1$, $B_2$, $B_1$,即:
\begin{displaymath}
	\begin{split}
		A_2 &= a_{n} a_{n-1} \ldots a_{\frac{n}{2} + 2} a_{\frac{n}{2} + 1}\\
		A_1 &= a_{\frac{n}{2}} a_{\frac{n}{2} - 1} \ldots a_2 a_1\\
		B_2 &= b_{n} b_{n-1} \ldots b_{\frac{n}{2} + 2} b_{\frac{n}{2} + 1}\\
		B_1 &= b_{\frac{n}{2}} b_{\frac{n}{2} - 1} \ldots b_2 b_1
	\end{split}
\end{displaymath}

则$A$可以写为$A = A_2 \times 2^{\frac{n}{2}} + A_1$.
$B$可以写为$B = B_2 \times 2^{\frac{n}{2}} + B_1$.
计算$A \times B$的问题在进行上述转换后表示为:
\begin{displaymath}
	\begin{split}
		A \times B
		& = (A_2 \times 2^{\frac{n}{2}} + A_1) \times (B_2 \times 2^{\frac{n}{2}} + B_1) \\
		& = A_2 B_2 \times 2^n + (A_2 B_1 + A_1 B_2) \times 2^{\frac{n}{2}} + A_1 B_1
	\end{split}
\end{displaymath}

此时将两个大数相乘的问题转化为4个乘法子问题和3个加法子问题。显然,分治策略还可以对子问题使用,继续减小问题的规模。

\subsection{伪代码}
\begin{algorithm}
	\DontPrintSemicolon{}
	\KwIn{Two large numbers $A$, $B$, which both have $n$ decimal digits}
	\KwResult{$A \times B$}
	\Begin{
		$n \leftarrow $ Number of Decimal Digits of $A$ and $B$\;
		\If{$n \neq 1$}{
			Divide $A$, $B$ into $A_2$, $A_1$, $B_2$ and $B_1$\;
			$C_3 \leftarrow DirectDAC(A_2, B_2)$\;
			$C_2 \leftarrow DirectDAC(A_2, B_1)$\;
			$C_1 \leftarrow DirectDAC(A_1, B_2)$\;
			$C_0 \leftarrow DirectDAC(A_1, B_1)$\;
			\KwRet{$C_3 \ll n + (C_2 + C_1) \ll (\frac{n}{2}) + C_0$}\;
		}
		\Else{
			\KwRet{$A \times B$}
		}
	}
	\caption{DirectDAC\label{label_for_pseudo_DirectDAC}}
\end{algorithm}

\subsection{复杂度分析}
由上述的算法描述可知,算法的主要开销来自于每次分支带来的4个乘法子问题和3个加法子问题,由于移位可在机器中由一个简单的指令完成,我们忽略这个操作的时间。\\
假设$T(n)$表示两个$n$位大数相乘所需的时间开销,则在直接分治法中:
\begin{displaymath}
	\begin{split}
		T(n)
		&= 4T(\frac{n}{2}) + 3n \\
		&= 4T(\frac{n}{2}) + O(n)
	\end{split}
\end{displaymath}

根据主方法,$\log_2 4  = 2> 1$, 推出如下定理:
\begin{theorem}{}{label_for_DirectDAC_complexity}
	用直接分治法计算$A \times B$的时间复杂度为$O(n^2)$.
\end{theorem}

根据定理\ref{thm:label_for_DirectDAC_complexity},直接分治法的性能是令人失望的,因为其并不能提供时间上优于直接相乘的性能。
但分治策略提示我们,这个算法的性能与乘法子问题的数目强相关。我们如果能够用一些其他的开销换取更少的乘法子问题数目,也许能得到更好的算法。


\newpage
\section{改进分治法}
\subsection{改进思路}
在直接分治法中,通过对大数进行分割,我们有:
\begin{displaymath}
	A \times B = A_2 B_2 \times 2^n + (A_2 B_1 + A_1 B_2) \times 2^{\frac{n}{2}} + A_1 B_1
\end{displaymath}

这个过程中,引入了4次乘法运算;在上一节中提到,分治策略和主定理提示我们尽可能减少乘法的次数。
但换取更低的乘法子问题数,需要其他的开销。
一种想法是,由于加法的复杂度为$O(n)$,我们也许可以用略多的加法子问题,来减少乘法子问题数。
基于此想法,我们对直接分治法作出一些改进。首先将直接分治法中的计算式修改为:
\begin{displaymath}
	\begin{split}
		A \times B
		&= A_2 B_2 \times 2^n + (A_2 B_1 + A_1 B_2) \times 2^{\frac{n}{2}} + A_1 B_1\\
		&= A_2 B_2 \times 2^n + ((A_2 + A_1)\times(B_2 + B_1) - A_2 B_2 - (A_1 B_1)) \times 2^{\frac{n}{2}} + A_1 B_1
	\end{split}
\end{displaymath}

观察上式,我们只需要做3次乘法,即计算$A_2 B_2$, $A_1 B_1$, $(A_2 + A_1)\times(B_2 + B_1)$, 以及4次加法,2次减法。
考虑到加法和减法本质上等同,我们成功地将这一问题转化为了3个乘法子问题和6个加法子问题。相比于直接分治法,我们降低了乘法的数量。

下面给出该算法的伪代码及复杂度分析。
\subsection{伪代码}
\begin{algorithm}
	\DontPrintSemicolon{}
	\KwIn{Two large numbers $A$, $B$, which both have $n$ decimal digits}
	\KwResult{$A \times B$}
	\Begin{
		$n \leftarrow $ Number of Decimal Digits of $A$ and $B$\;
		\If{$n \neq 1$}{
			Divide $A$, $B$ into $A_2$, $A_1$, $B_2$ and $B_1$\;
			$C_2 \leftarrow DirectDAC(A_2, B_2)$\;
			$C_1 \leftarrow DirectDAC(A_1, B_1)$\;
			$C_0 \leftarrow DirectDAC(A_2 + A_1, B_2 + B_1)$\;
			\KwRet{$C_2 \ll n + (C_0 - C_2 - C_1) \ll (\frac{n}{2}) + C_1$}\;
		}
		\Else{
			\KwRet{$A \times B$}
		}
	}
	\caption{ModifiedDAC\label{label_for_pseudo_ModifiedDAC}}
\end{algorithm}

\subsection{复杂度分析}
同上节的复杂度分析,我们此处也忽略移位操作带来的开销。改进分治法中,我们将问题分解为3个乘法子问题与6个加法子问题。
因此有:
\begin{displaymath}
	\begin{split}
		T(n)
		&= 3T(\frac{n}{2}) + 6n\\
		&= 3T(\frac{n}{2}) + O(n)
	\end{split}
\end{displaymath}

根据主方法,$\log_2 3 > 1$. 推出如下定理:
\begin{theorem}{}{label_for_ModifiedDAC_complexity}
	用改进分治法计算$A \times B$的时间复杂度为$O(n^{\log_2 3}) \approx O(n^{1.585})$.
\end{theorem}
